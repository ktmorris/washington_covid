% Options for packages loaded elsewhere
\PassOptionsToPackage{unicode}{hyperref}
\PassOptionsToPackage{hyphens}{url}
%
\documentclass[
  ignorenonframetext,
  aspectratio=169]{beamer}
\usepackage{pgfpages}
\setbeamertemplate{caption}[numbered]
\setbeamertemplate{caption label separator}{: }
\setbeamercolor{caption name}{fg=normal text.fg}
\beamertemplatenavigationsymbolsempty
% Prevent slide breaks in the middle of a paragraph
\widowpenalties 1 10000
\raggedbottom
\setbeamertemplate{part page}{
  \centering
  \begin{beamercolorbox}[sep=16pt,center]{part title}
    \usebeamerfont{part title}\insertpart\par
  \end{beamercolorbox}
}
\setbeamertemplate{section page}{
  \centering
  \begin{beamercolorbox}[sep=12pt,center]{part title}
    \usebeamerfont{section title}\insertsection\par
  \end{beamercolorbox}
}
\setbeamertemplate{subsection page}{
  \centering
  \begin{beamercolorbox}[sep=8pt,center]{part title}
    \usebeamerfont{subsection title}\insertsubsection\par
  \end{beamercolorbox}
}
\AtBeginPart{
  \frame{\partpage}
}
\AtBeginSection{
  \ifbibliography
  \else
    \frame{\sectionpage}
  \fi
}
\AtBeginSubsection{
  \frame{\subsectionpage}
}
\usepackage{amsmath,amssymb}
\usepackage{lmodern}
\usepackage{ifxetex,ifluatex}
\ifnum 0\ifxetex 1\fi\ifluatex 1\fi=0 % if pdftex
  \usepackage[T1]{fontenc}
  \usepackage[utf8]{inputenc}
  \usepackage{textcomp} % provide euro and other symbols
\else % if luatex or xetex
  \usepackage{unicode-math}
  \defaultfontfeatures{Scale=MatchLowercase}
  \defaultfontfeatures[\rmfamily]{Ligatures=TeX,Scale=1}
\fi
\usetheme[]{Berlin}
% Use upquote if available, for straight quotes in verbatim environments
\IfFileExists{upquote.sty}{\usepackage{upquote}}{}
\IfFileExists{microtype.sty}{% use microtype if available
  \usepackage[]{microtype}
  \UseMicrotypeSet[protrusion]{basicmath} % disable protrusion for tt fonts
}{}
\makeatletter
\@ifundefined{KOMAClassName}{% if non-KOMA class
  \IfFileExists{parskip.sty}{%
    \usepackage{parskip}
  }{% else
    \setlength{\parindent}{0pt}
    \setlength{\parskip}{6pt plus 2pt minus 1pt}}
}{% if KOMA class
  \KOMAoptions{parskip=half}}
\makeatother
\usepackage{xcolor}
\IfFileExists{xurl.sty}{\usepackage{xurl}}{} % add URL line breaks if available
\IfFileExists{bookmark.sty}{\usepackage{bookmark}}{\usepackage{hyperref}}
\hypersetup{
  pdftitle={Grief and Anger},
  pdfauthor={Kevin Morris},
  hidelinks,
  pdfcreator={LaTeX via pandoc}}
\urlstyle{same} % disable monospaced font for URLs
\newif\ifbibliography
\setlength{\emergencystretch}{3em} % prevent overfull lines
\providecommand{\tightlist}{%
  \setlength{\itemsep}{0pt}\setlength{\parskip}{0pt}}
\setcounter{secnumdepth}{-\maxdimen} % remove section numbering
\ifluatex
  \usepackage{selnolig}  % disable illegal ligatures
\fi
\newlength{\cslhangindent}
\setlength{\cslhangindent}{1.5em}
\newlength{\csllabelwidth}
\setlength{\csllabelwidth}{3em}
\newenvironment{CSLReferences}[2] % #1 hanging-ident, #2 entry spacing
 {% don't indent paragraphs
  \setlength{\parindent}{0pt}
  % turn on hanging indent if param 1 is 1
  \ifodd #1 \everypar{\setlength{\hangindent}{\cslhangindent}}\ignorespaces\fi
  % set entry spacing
  \ifnum #2 > 0
  \setlength{\parskip}{#2\baselineskip}
  \fi
 }%
 {}
\usepackage{calc}
\newcommand{\CSLBlock}[1]{#1\hfill\break}
\newcommand{\CSLLeftMargin}[1]{\parbox[t]{\csllabelwidth}{#1}}
\newcommand{\CSLRightInline}[1]{\parbox[t]{\linewidth - \csllabelwidth}{#1}\break}
\newcommand{\CSLIndent}[1]{\hspace{\cslhangindent}#1}

\title{Grief and Anger}
\subtitle{Disentangling the Turnout Effects of COVID-19}
\author{Kevin Morris}
\date{Annual Meeting of the American Political Science Association,
10/02/2021}
\institute{Brennan Center for Justice}

\begin{document}
\frame{\titlepage}
\begin{abstract}
\href{mailto:kevin.morris@nyu.edu}{\nolinkurl{kevin.morris@nyu.edu}}
\end{abstract}

\begin{frame}{COVID-19 and the Presidential Election Were Joined at the
Hip}
\protect\hypertarget{covid-19-and-the-presidential-election-were-joined-at-the-hip}{}
\begin{itemize}[<+->]
\tightlist
\item
  By the time of the election, more than \textbf{8.3 million} Americans
  had tested positive
\end{itemize}

\begin{itemize}[<+->]
\tightlist
\item
  and more than \textbf{220 thousand} had died from the virus.
\end{itemize}

\begin{itemize}[<+->]
\tightlist
\item
  Importantly, there were big---and widely known---racial disparities in
  the effects of the pandemic.
\end{itemize}
\end{frame}

\begin{frame}{COVID-19 and the Presidential Election Were Joined at the
Hip}
\protect\hypertarget{covid-19-and-the-presidential-election-were-joined-at-the-hip-1}{}
\end{frame}

\begin{frame}{Political Threat}
\protect\hypertarget{political-threat}{}
\begin{itemize}[<+->]
\tightlist
\item
  A large body of research demonstrates that when the government makes
  life harder for citizens, they can be mobilized to participate.
\end{itemize}

\begin{itemize}[<+->]
\tightlist
\item
  Tam Cho, Gimpel, and Wu (2006) shows that Arab-Americans who face a
  worrisome policy environment can be mobilized.
\end{itemize}

\begin{itemize}[<+->]
\tightlist
\item
  White (2016) argues that more aggressive immigration enforcement can
  increase turnout among Latinos.
\end{itemize}

\begin{itemize}[<+->]
\tightlist
\item
  Campbell (2003) documents that threats to Social Security increased
  political activity among seniors.
\end{itemize}
\end{frame}

\begin{frame}{Political Threat}
\protect\hypertarget{political-threat-1}{}
\begin{itemize}[<+->]
\tightlist
\item
  Nichols and Valdéz (2020): ``The straightforward notion that
  racialized threat mobilizes Latinxs is weakened by the literature's
  tendency to overpredict political mobilization, overlook individual
  responses to threat, and disregard the role of mobilization
  structures.''
\end{itemize}
\end{frame}

\begin{frame}{Opportunity Cost}
\protect\hypertarget{opportunity-cost}{}
\begin{itemize}[<+->]
\tightlist
\item
  All the way back to Rosenstone (1982), who shows that economic
  hardship reduces participation\ldots{}
\end{itemize}

\begin{itemize}[<+->]
\tightlist
\item
  \ldots up through contemporary work showing that stressful life events
  such as divorce (Sandell and Plutzer 2005) and ill health (Pacheco and
  Fletcher 2015) can decrease turnout too.
\end{itemize}

\begin{itemize}[<+->]
\tightlist
\item
  Hobbs, Christakis, and Fowler (2014) is among the only papers to
  directly study the effect of a familial death on turnout by matching
  Social Security death records to the California registered voter file.
  They find that a household death around an election reduces turnout.
\end{itemize}
\end{frame}

\begin{frame}{From Demobilizing to Mobilizing}
\protect\hypertarget{from-demobilizing-to-mobilizing}{}
\begin{itemize}[<+->]
\tightlist
\item
  The racial injustice narrative, borrowed from the criminal justice
  literature, indicates that historically demobilizing events can be
  mobilizing when located in a narrative of racial injustice (Walker
  2020).
\end{itemize}

\begin{itemize}[<+->]
\tightlist
\item
  COVID-19 had the potential to bring such a narrative to life given A)
  the racial disparities in death (e.g. Garg et al. 2020) and B) the
  intense reporting of these disparities in both the mainstream press
  (Kolata 2020) and publications specifically serving Black America
  (Blount 2020).
\end{itemize}

\begin{itemize}[<+->]
\tightlist
\item
  Racial minorities with a strong sense of ``racial affinity'' or
  ``linked fate'' are probably more likely to understand COVID-19 and a
  \emph{racial} policy failure and therefore be mobilized.
\end{itemize}
\end{frame}

\begin{frame}{Competing Treatments}
\protect\hypertarget{competing-treatments}{}
\begin{center}\includegraphics[width=0.9\linewidth,height=0.9\textheight]{../../temp/dir} \end{center}
\end{frame}

\begin{frame}{Competing Treatments}
\protect\hypertarget{competing-treatments-1}{}
\begin{center}\includegraphics[width=0.9\linewidth,height=0.9\textheight]{../../temp/dir} \end{center}
\end{frame}

\begin{frame}{Competing Treatments}
\protect\hypertarget{competing-treatments-2}{}
\begin{center}\includegraphics[width=0.9\linewidth,height=0.9\textheight]{../../temp/dir} \end{center}
\end{frame}

\begin{frame}{Competing Treatments}
\protect\hypertarget{competing-treatments-3}{}
\begin{center}\includegraphics[width=0.9\linewidth,height=0.9\textheight]{../../temp/dir} \end{center}
\end{frame}

\begin{frame}{Methods and Hypotheses}
\protect\hypertarget{methods-and-hypotheses}{}
\begin{itemize}[<+->]
\tightlist
\item
  Create groups of three, matched from the ``treated'' voter:
\end{itemize}

\begin{enumerate}[<+->]
\tightlist
\item
  Both treatments (Household Death + Household Death from COVID)
\end{enumerate}

\begin{enumerate}[<+->]
\setcounter{enumi}{1}
\tightlist
\item
  One treatment (Household Death)
\end{enumerate}

\begin{enumerate}[<+->]
\setcounter{enumi}{2}
\tightlist
\item
  No treatment
\end{enumerate}
\end{frame}

\begin{frame}{Methods and Hypotheses}
\protect\hypertarget{methods-and-hypotheses-1}{}
\begin{itemize}[<+->]
\tightlist
\item
  While administrative data is very helpful, it doesn't really give us
  access to the causal mechanism.
\end{itemize}

\begin{itemize}[<+->]
\tightlist
\item
  I use CCES data to explore whether the turnout effects of COVID-19
  contact were moderated by non-white Americans' views on structural
  racism using a ``racial affinity'' index.
\end{itemize}

\begin{itemize}[<+->]
\tightlist
\item
  \emph{H}: Non-white voters with higher levels of racial affinity were
  more mobilized by COVID-19 contact than voters without such affinity.
\end{itemize}
\end{frame}

\begin{frame}{Results}
\protect\hypertarget{results}{}
\begin{center}
?
\end{center}
\end{frame}

\begin{frame}{Results}
\protect\hypertarget{results-1}{}
\begin{center}
???
\end{center}
\end{frame}

\begin{frame}{Results}
\protect\hypertarget{results-2}{}
\begin{center}
?????
\end{center}
\end{frame}

\begin{frame}{(Here's what this could look like)}
\protect\hypertarget{heres-what-this-could-look-like}{}
\begin{center}\includegraphics[width=0.65\linewidth,height=0.8\textheight]{../../temp/example} \end{center}
\end{frame}

\begin{frame}{Survey Data: CCES}
\protect\hypertarget{survey-data-cces}{}
\begin{center}\includegraphics[width=0.8\linewidth,height=0.8\textheight]{../../temp/example2} \end{center}
\end{frame}

\begin{frame}{Survey Data: CCES}
\protect\hypertarget{survey-data-cces-1}{}
\begin{center}\includegraphics[width=0.8\linewidth,height=0.8\textheight]{../../temp/example3} \end{center}
\end{frame}

\begin{frame}{Survey Data: CCES}
\protect\hypertarget{survey-data-cces-2}{}
\begin{center}\includegraphics[width=0.8\linewidth,height=0.8\textheight]{../../temp/example4} \end{center}
\end{frame}

\begin{frame}{Survey Data: CCES}
\protect\hypertarget{survey-data-cces-3}{}
\begin{center}\includegraphics[width=0.8\linewidth,height=0.8\textheight]{../../temp/example5} \end{center}
\end{frame}

\begin{frame}{Takeaways}
\protect\hypertarget{takeaways}{}
\begin{itemize}[<+->]
\tightlist
\item
  When we look at the effect of COVID-19 contact on turnout in the
  \emph{aggregate}, there doesn't seem to be much of a story.
\end{itemize}

\begin{itemize}[<+->]
\tightlist
\item
  But when we explore how this ``treatment effect'' is moderated by
  views about structural racism in the United States, big gaps open up
  in how nonwhite voters responded to being touched by the pandemic.
\end{itemize}

\begin{itemize}[<+->]
\tightlist
\item
  When we have treatment effects (potentially) pointing in different
  directions, we should develop causal models that can disentangle these
  effects rather than simply adjudicate between competing hypotheses.
\end{itemize}

\begin{itemize}[<+->]
\tightlist
\item
  The IRB can be annoying.
\end{itemize}
\end{frame}

\begin{frame}{Thanks!}
\protect\hypertarget{thanks}{}
\href{mailto:kevin.morris@nyu.edu}{\nolinkurl{kevin.morris@nyu.edu}}

\(@\)KevinTMorris
\end{frame}

\begin{frame}[allowframebreaks]{References}
\protect\hypertarget{references}{}
\hypertarget{refs}{}
\begin{CSLReferences}{1}{0}
\leavevmode\hypertarget{ref-Blount2020}{}%
Blount, Linda Goler. 2020. {``Fighting {Injustice} in {Health Care} for
{Black Women} and {Diverse Rare Disease Patients}.''} \emph{Ebony}, July
29, 2020.
\url{https://www.ebony.com/health/fighting-injustice-in-health-care-for-black-women-and-diverse-rare-disease-patients/}.

\leavevmode\hypertarget{ref-Campbell2003a}{}%
Campbell, Andrea Louise. 2003. {``Participatory {Reactions} to {Policy
Threats}: Senior {Citizens} and the {Defense} of {Social Security} and
{Medicare}.''} \emph{Political Behavior} 25 (1): 29--49.
\url{https://doi.org/10.1023/A:1022900327448}.

\leavevmode\hypertarget{ref-Garg2020}{}%
Garg, Shikha, Lindsay Kim, Michael Whitaker, Alissa O'Halloran, Charisse
Cummings, Rachel Holstein, Mila Prill, et al. 2020. {``Hospitalization
{Rates} and {Characteristics} of {Patients Hospitalized} with
{Laboratory}-{Confirmed Coronavirus Disease} 2019 --- {COVID}-{NET}, 14
{States}, {March} 1--30, 2020.''} \emph{MMWR. Morbidity and Mortality
Weekly Report} 69 (15): 458--64.
\url{https://doi.org/10.15585/mmwr.mm6915e3}.

\leavevmode\hypertarget{ref-Hobbs2014}{}%
Hobbs, William R., Nicholas A. Christakis, and James H. Fowler. 2014.
{``Widowhood {Effects} in {Voter Participation}.''} \emph{American
Journal of Political Science} 58 (1): 1--16.
\url{https://doi.org/10.1111/ajps.12040}.

\leavevmode\hypertarget{ref-Kolata2020}{}%
Kolata, Gina. 2020. {``Social {Inequities Explain Racial Gaps} in
{Pandemic}, {Studies Find}.''} \emph{The New York Times: Health},
December 9, 2020.
\url{https://www.nytimes.com/2020/12/09/health/coronavirus-black-hispanic.html}.

\leavevmode\hypertarget{ref-Nichols2020a}{}%
Nichols, Vanessa Cruz, and Ramón Garibaldo Valdéz. 2020. {``How to
{Sound} the {Alarms}: Untangling {Racialized Threat} in {Latinx
Mobilization}.''} \emph{PS: Political Science \& Politics} 53 (4):
690--96. \url{https://doi.org/10.1017/S1049096520000530}.

\leavevmode\hypertarget{ref-Pacheco2015}{}%
Pacheco, Julianna, and Jason Fletcher. 2015. {``Incorporating {Health}
into {Studies} of {Political Behavior}: Evidence for {Turnout} and
{Partisanship}.''} \emph{Political Research Quarterly} 68 (1): 104--16.
\url{https://doi.org/10.1177/1065912914563548}.

\leavevmode\hypertarget{ref-Rosenstone1982}{}%
Rosenstone, Steven J. 1982. {``Economic {Adversity} and {Voter
Turnout}.''} \emph{American Journal of Political Science} 26 (1):
25--46. \url{https://doi.org/10.2307/2110837}.

\leavevmode\hypertarget{ref-Sandell2005}{}%
Sandell, Julianna, and Eric Plutzer. 2005. {``Families, Divorce and
Voter Turnout in the {US}.''} \emph{Political Behavior} 27 (2): 133--62.
\url{https://doi.org/10.1007/s11109-005-3341-9}.

\leavevmode\hypertarget{ref-TamCho2006a}{}%
Tam Cho, Wendy K., James G. Gimpel, and Tony Wu. 2006. {``Clarifying the
{Role} of {SES} in {Political Participation}: Policy {Threat} and {Arab
American Mobilization}.''} \emph{Journal of Politics} 68 (4): 977--91.
\url{https://doi.org/10.1111/j.1468-2508.2006.00482.x}.

\leavevmode\hypertarget{ref-Walker2020}{}%
Walker, Hannah L. 2020. {``Targeted: The {Mobilizing Effect} of
{Perceptions} of {Unfair Policing Practices}.''} \emph{The Journal of
Politics} 82 (1): 119--34. \url{https://doi.org/10.1086/705684}.

\leavevmode\hypertarget{ref-White2016}{}%
White, Ariel. 2016. {``When {Threat Mobilizes}: Immigration
{Enforcement} and {Latino Voter Turnout}.''} \emph{Political Behavior}
38 (2): 355--82. \url{https://doi.org/10.1007/s11109-015-9317-5}.

\end{CSLReferences}
\end{frame}

\end{document}
