% Options for packages loaded elsewhere
\PassOptionsToPackage{unicode}{hyperref}
\PassOptionsToPackage{hyphens}{url}
%
\documentclass[
  12pt,
]{article}
\usepackage{lmodern}
\usepackage{amssymb,amsmath}
\usepackage{ifxetex,ifluatex}
\ifnum 0\ifxetex 1\fi\ifluatex 1\fi=0 % if pdftex
  \usepackage[T1]{fontenc}
  \usepackage[utf8]{inputenc}
  \usepackage{textcomp} % provide euro and other symbols
\else % if luatex or xetex
  \usepackage{unicode-math}
  \defaultfontfeatures{Scale=MatchLowercase}
  \defaultfontfeatures[\rmfamily]{Ligatures=TeX,Scale=1}
\fi
% Use upquote if available, for straight quotes in verbatim environments
\IfFileExists{upquote.sty}{\usepackage{upquote}}{}
\IfFileExists{microtype.sty}{% use microtype if available
  \usepackage[]{microtype}
  \UseMicrotypeSet[protrusion]{basicmath} % disable protrusion for tt fonts
}{}
\makeatletter
\@ifundefined{KOMAClassName}{% if non-KOMA class
  \IfFileExists{parskip.sty}{%
    \usepackage{parskip}
  }{% else
    \setlength{\parindent}{0pt}
    \setlength{\parskip}{6pt plus 2pt minus 1pt}}
}{% if KOMA class
  \KOMAoptions{parskip=half}}
\makeatother
\usepackage{xcolor}
\IfFileExists{xurl.sty}{\usepackage{xurl}}{} % add URL line breaks if available
\IfFileExists{bookmark.sty}{\usepackage{bookmark}}{\usepackage{hyperref}}
\hypersetup{
  pdftitle={Grief and Anger: Disentangling the Effects of COVID-19 on Turnout},
  pdfauthor={Kevin Morris},
  hidelinks,
  pdfcreator={LaTeX via pandoc}}
\urlstyle{same} % disable monospaced font for URLs
\usepackage[margin=1in]{geometry}
\usepackage{longtable,booktabs}
% Correct order of tables after \paragraph or \subparagraph
\usepackage{etoolbox}
\makeatletter
\patchcmd\longtable{\par}{\if@noskipsec\mbox{}\fi\par}{}{}
\makeatother
% Allow footnotes in longtable head/foot
\IfFileExists{footnotehyper.sty}{\usepackage{footnotehyper}}{\usepackage{footnote}}
\makesavenoteenv{longtable}
\usepackage{graphicx}
\makeatletter
\def\maxwidth{\ifdim\Gin@nat@width>\linewidth\linewidth\else\Gin@nat@width\fi}
\def\maxheight{\ifdim\Gin@nat@height>\textheight\textheight\else\Gin@nat@height\fi}
\makeatother
% Scale images if necessary, so that they will not overflow the page
% margins by default, and it is still possible to overwrite the defaults
% using explicit options in \includegraphics[width, height, ...]{}
\setkeys{Gin}{width=\maxwidth,height=\maxheight,keepaspectratio}
% Set default figure placement to htbp
\makeatletter
\def\fps@figure{htbp}
\makeatother
\setlength{\emergencystretch}{3em} % prevent overfull lines
\providecommand{\tightlist}{%
  \setlength{\itemsep}{0pt}\setlength{\parskip}{0pt}}
\setcounter{secnumdepth}{5}
\usepackage{rotating}
\usepackage{setspace}
\usepackage{lineno}
\usepackage{booktabs}
\usepackage{longtable}
\usepackage{array}
\usepackage{multirow}
\usepackage{wrapfig}
\usepackage{float}
\usepackage{colortbl}
\usepackage{pdflscape}
\usepackage{tabu}
\usepackage{threeparttable}
\usepackage{threeparttablex}
\usepackage[normalem]{ulem}
\usepackage{makecell}
\usepackage{xcolor}
\newlength{\cslhangindent}
\setlength{\cslhangindent}{1.5em}
\newenvironment{cslreferences}%
  {\setlength{\parindent}{0pt}%
  \everypar{\setlength{\hangindent}{\cslhangindent}}\ignorespaces}%
  {\par}

\title{Grief and Anger: Disentangling the Effects of COVID-19 on Turnout\thanks{Thanks.}}
\author{Kevin Morris\footnote{Researcher, Brennan Center for Justice at NYU School of Law, 120 Broadway Ste 1750, New York, NY 10271 (\href{mailto:kevin.morris@nyu.edu}{\nolinkurl{kevin.morris@nyu.edu}})}}
\date{October 22, 2020}

\begin{document}
\maketitle
\begin{abstract}
The novel coronavirus SARS-CoV-2 (COVID-19) has upended normal routines across the United States. More than 150 thousand Americans have died of COVID-19, and more than 5 million have tested positive for the illness. At the same time, there is widespread perception that the government's response to the pandemic has been lackluster. This manuscript tests whether the coronavirus has been mobilizing for voters most directly impacted by pandemic. By leveraging individual-level death records from Washington State, we identify registered voters who lived with individuals who died from COVID-19, measuring their 2020 turnout against voters who lived with individuals who died from other causes, and voters who did not live with anyone who died. These analyses are supplemented with survey data exploring whether COVID-19 was differently mobilizing for voters who believed the government mishandled the public response.
\end{abstract}

\pagenumbering{gobble}
\pagebreak

\pagenumbering{arabic}

\hypertarget{introduction}{%
\subsection*{Introduction}\label{introduction}}
\addcontentsline{toc}{subsection}{Introduction}

On January 21, 2020, the first case of SARS-CoV-2 (or ``COVID-19'') was confirmed in the state of Washington (McNerthney \protect\hyperlink{ref-McNerthney2020}{2020}). President Donald Trump made his first public comments on the new virus the next day from Davos, telling CNBC ``we have it totally under control. It's one person coming in from China, and we have it under control'' (Calia \protect\hyperlink{ref-Calia2020}{2020}). A few days later, on January 30, he said at a manufacturing plant in Michigan that ``we think we have it very well under control,'' assuring listeners that ``it's going to have a very good ending for it'' (``Remarks by President Trump at a USMCA Celebration with American Workers \textbar{} Warren, MI'' \protect\hyperlink{ref-whitehouse2020}{2020}). Meanwhile, cases were identified in states across the country. The first confirmed death attributed to COVID-19 occurred in Northern California on February 4th (Moon \protect\hyperlink{ref-Moon2020}{2020}). By early March, Washington State was being called the center of the outbreak in the United States (Golden \protect\hyperlink{ref-Golden2020}{2020}), although New York City would soon claim that dubious honor. By the time of the 2020 presidential election, more than 8.3 million Americans had tested positive for the novel coronavirus, with more than 220 thousand dead (\emph{New York Times: U.S.} \protect\hyperlink{ref-nyt2020}{2020}). Reporting from a few months earlier, however, indicates that the official reports may be undercounts (Lu \protect\hyperlink{ref-Lu2020}{2020}).

Although the World Health Organization declared COVID-19 a pandemic on March 11 (Wan \protect\hyperlink{ref-Wan2020}{2020}), the response from the United State's federal government was slow. In the months to come, the Trump Administration would push responsibility to states and local governments, downplay the severity of the pandemic, and argue against the widespread stay-at-home orders promoted by public health experts (Shear et al. \protect\hyperlink{ref-Shear2020}{2020}). Much of the United States remained at home throughout the summer even as peer nations were able to return to more normal daily life (Douglas \protect\hyperlink{ref-Douglas2020}{2020}). The Trump Administration's response to the pandemic has consistently been criticized in the press, with Time Magazine reporting that a "complete catalog of Trump's failures to adequately address the pandemic is the stuff of books, not single articles' (Fitzpatrick \protect\hyperlink{ref-Fitzpatrick2020}{2020}).

These extraordinary upheavals to everyday life occurred against an impending presidential election in the United States. On November 3, 2020, Americans cast ballots deciding whether to re-elect president Trump or rather election Joseph Biden. Understanding how the pandemic shaped voters' behavior and preferences is of key importance in the aftermath of such a historic contest, and this manuscript sheds light on how personal contact with the novel coronavirus impacted voters. Using individual-level death records from the state of Washington, we identify voters who lived with individuals who perished from COVID-19. We compare their turnout to the turnout of voters who lived with individuals who died from other causes, as well as voters who did not live with anyone who died. We then look to national survey data to understand if the salience of the pandemic and respondents' beliefs about the federal government's response was associated with voter turnout.

\hypertarget{literature-and-theory}{%
\subsection*{Literature and Theory}\label{literature-and-theory}}
\addcontentsline{toc}{subsection}{Literature and Theory}

We begin by considering the expected effect of the death of a household member in the context of the opportunity cost literature. Although there is little existing literature relating the exogenous shock of the death of a loved one to political participation (but see Hobbs, Christakis, and Fowler \protect\hyperlink{ref-Hobbs2014}{2014}), political scientists have long noted that voters vote at lower rates when faced with competing demands on their time. After exploring the impact of economic adversity on turnout, Rosenstone (\protect\hyperlink{ref-Rosenstone1982}{1982}, 42, emphasis added) argues that ``theories of political participation must take into account not only the costs associated specifically with the act of participation but other life circumstances that also place demands on the citizen\ldots{} {[}conditions{]} such as marital problems, \emph{family illness, death of a close friend or relative}, problems at work, or a change of residences also surely increase the opportunity costs of political participation.'' Caring for a sick loved one, as well as funereal costs, impose large financial burdens on households (e.g.~Fan and Zick \protect\hyperlink{ref-Fan2004}{2004}). Wyatt et al. (\protect\hyperlink{ref-Wyatt1999}{1999}) studied caregivers after the person for whom they were caring died, finding that these caregivers spent an average of 10.8 hours per day on direct care.

Hobbs, Christakis, and Fowler (\protect\hyperlink{ref-Hobbs2014}{2014}) is unique in its investigation of the direct, short-term effects of the death of a spouse on voter turnout. They link records from the Social Security Master Death File to the California registered voter file to identify voters whose spouses had died in the months before and after the California Special Statewide 2009, Gubernatorial Primary 2010, and Gubernatorial General 2010 elections. Unsurprisingly, they find that voters whose spouses die immediately before or after an election are substantially less likely to vote than their matches controls, whose spouses did not die.

Although little research has taken up Rosenstone's charge to explore the effects of family illness or death on voter turnout, the research on other types of household shocks is perhaps illustrative. There is a large body of literature indicating that economic shocks, such as the loss of a job, cause lower turnout, especially when these shocks are not linked to wider social circumstances (Emmenegger, Marx, and Schraff \protect\hyperlink{ref-Emmenegger2015}{2015}; Marx and Nguyen \protect\hyperlink{ref-Marx2016}{2016}; Margalit \protect\hyperlink{ref-Margalit2019}{2019}). Hassell and Settle (\protect\hyperlink{ref-Hassell2017}{2017}) argues that voters without a strong history of participation disengage from political life when faced with stressful situations, though individuals with a robust history of engagement are unaffected. Negative health events are also linked with lower turnout (Pacheco and Fletcher \protect\hyperlink{ref-Pacheco2015}{2015}); so too is divorce (Sandell and Plutzer \protect\hyperlink{ref-Sandell2005}{2005}). Each of these stressful life events demand a great deal of time, financial resources, and energy --- resources that cannot then be used to register to vote, to learn about candidates, or to locate one's polling place.

There are, however, circumstances in which adverse life conditions can mobilize voters: namely, when they link their conditions to broader societal issues and perceive a political threat. Piven and Cloward (\protect\hyperlink{ref-Piven1979}{1979}) and Schlozman and Verba (\protect\hyperlink{ref-Schlozman1979}{1979}), for instance, document how workers become politically engaged when they understand their unemployment as a widespread phenomenon that demands collective government response. Cho, Gimpel, and Wu (\protect\hyperlink{ref-Cho2006a}{2006}, 978) explains: ``A solid body of evidence\ldots{} indicates that political mobilization is a direct response to the degree of threat and discrimination a group experiences.'' Cho, Gimpel, and Wu (\protect\hyperlink{ref-Cho2006a}{2006}) examines the impact of Arab Americans' response to ``worrisome government policy actions,'' finding that citizens with greater access to political information were mobilized by these threats. Gutierrez et al. (\protect\hyperlink{ref-Gutierrez2019}{2019}) explores how racial threat mobilized Latinos in the 2016 presidential election, while White (\protect\hyperlink{ref-White2016}{2016}) demonstrates that increased immigration enforcement at the county-level increased Latinos' participation as well. The phenomenon is not limited to threats based on racial identity or joblessness: Campbell (\protect\hyperlink{ref-Campbell2003}{2003}), for instance, documents a surge in political letter writing among senior citizens in response to threats to Social Security and Medicare. There is some indication that the mobilization of a policy threat is particularly pronounced among voters with high personal efficacy (Rudolph, Gangl, and Stevens \protect\hyperlink{ref-Rudolph2000}{2000}).

Walker (\protect\hyperlink{ref-Walker2020}{2020}) offers key insight into how a traditionally demobilizing experience can instead become a force for mobilization. She examines the effects of contact with the criminal justice system on political participation, interactions historically understood as reducing participation at all levels (see, for instance, Weaver and Lerman \protect\hyperlink{ref-Weaver2010}{2010}; Lerman and Weaver \protect\hyperlink{ref-Lerman2014}{2014}). Walker finds, however, that the demobilizing effect of contact with the criminal justice system is moderated by how an individual locates their experience in the context of national policy. She introduces the injustice framework, arguing that when individuals believe that their criminal justice involvement arises from unjust systemic forces, they can be mobilized to undertake political actions.

The political threat literature and Walker's injustice framework sheds light on the potential mechanism through which the novel coronavirus might increase turnout among voters most directly exposed. The racially disparate nature of the pandemic coupled with the widespread perception that the federal government did not take it seriously seems ripe for interpretations of unfairness and threat. The rhetoric widely used in the American press was that the damage inflicted by COVID-19 was thanks largely to political failures. ``The Deaths of 150,000 Americans Are on Trump's Hands,'' read a headline in The Nation (Mystal \protect\hyperlink{ref-Mystal2020}{2020}). David Frum at the Atlantic called the pandemic ``Trump's Plague'' (Frum \protect\hyperlink{ref-Frum2020}{2020}) and Slate ran an article titled ``How Trump Killed Tens of Thousands of Americans'' (Saletan \protect\hyperlink{ref-Saletan2020}{2020}). ``Can We Call Trump a Killer?'' asked Blow (\protect\hyperlink{ref-Blow2020}{2020}) in the headline of a column at the New York Times. It would seem that, if a voter answered that question in the affirmative after losing a loved one to the pandemic, she might be mobilized to vote against the incumbent --- despite the disruptive nature of living with a loved one who died.

By comparing the turnout of voters who lived with individuals who died from COVID-19 to both voters who lived with individuals who perished from other causes \emph{and} voters who lived with no one who died, this manuscript estimates the net effect of the pandemic on turnout, and the relative magnitude of the (depressive) opportunity-cost and (mobilizing) political-threat treatments. By using survey data to interrogate how the salience of COVID-19 and perceived missteps by the federal government interact to structure turnout, we also get insight into the causal mechanisms at play.

\hypertarget{research-design-and-data}{%
\subsection*{Research Design and Data}\label{research-design-and-data}}
\addcontentsline{toc}{subsection}{Research Design and Data}

Registered voters who lived with household members who died from COVID-19 received, in effect, two treatments. On the one hand, they faced an extreme upheaval in the death of a household member. As discussed above, there is evidence suggesting that such household-level shocks can reduce turnout. At the same time, however, these individuals were also deeply exposed to what many consider a massive policy failure --- the government's response to the pandemic. The political threat literature indicates that such exposure could be mobilizing.

\hypertarget{double-matched-triple-differences}{%
\subsubsection*{Double-Matched Triple-Differences}\label{double-matched-triple-differences}}
\addcontentsline{toc}{subsubsection}{Double-Matched Triple-Differences}

The primary data used for the initial section of this manuscript are administrative records from the State of Washington. We begin by identifying all individuals who died between January 1 and November 3, 2020.\footnote{See \url{https://www.doh.wa.gov/DataandStatisticalReports/HealthStatistics/Death}} These records include a variety of data about the deceased individual, including residential latitude and longitude and cause of death.

We then find all registered voters who lived with someone who died during this period. The geocoded registered voter file comes from L2 Political, and includes a variety of characteristics such as age, party affiliation, gender, turnout history, as well as estimates of voters' race and ethnicity. Using the voter file we identify all voters who lived within 5 meters of an individual who died. Following White (\protect\hyperlink{ref-White2019a}{2019}), which similarly matched administrative records to the registered voter file, we exclude addresses at which more than 10 voters are registered. This avoids labeling all voters of apartment buildings or nursing homes as treated.

We use a difference-in-differences-in-differences, or triple-differences, approach. We compare the turnout of individuals whose housemates had died from COVID-19 to housemates of voters who died for other reasons, and to the turnout of individuals who did not live with someone who died. There are real reasons to suspect that these three groups varied in important ways. We expect that older voters are more likely to live with an individual who died during the time period. One way to control for group-level differences is to use a variety of covariates in a traditional difference-in-differences framework, though that runs the risk of violating the parallel trends assumption. Here, we adopt a different approach: prior to implementing the difference-in-differences model we match voters using a genetic matching algorithm (Sekhon \protect\hyperlink{ref-Sekhon2011}{2011}), thereby ensuring that our treatment and control groups are substantially similar prior to the testing for treatment effects.

In the case of the study at hand, we specifically use what we call a double-matched triple-differences framework. We begin by matching each treated voter (a registered voter who lived with someone who died from COVID-19) to a primary control voter (a registered voter who lived with someone who died from something other than COVID-19). We assume that the opportunity-cost effect of living with a loved one who died was the same for both groups. By testing for a turnout difference between these groups, we therefore measure the political threat effects unique to COVID-19.

This simple set-up does not allow us to test the \emph{overall} effect of proximal contact with COVID-19. To find the net effect, we must also compare treated voters with voters who lived with no one who died (secondary control voters). Treated voters and primary control voters are each matched to a secondary control voter using the same characteristics as the initial match. We thus construct groups of four voters: the treated voter; the primary control voter; and two secondary control voters. Any difference between the primary control and secondary control voters' turnout will measure the opportunity cost effect of living with an individual who died. Any difference between the treated and primary control voters measures the political threat mobilization associated with deaths arising from COVID-19.

Given that Hobbs, Christakis, and Fowler (\protect\hyperlink{ref-Hobbs2014}{2014}) indicates that the timing of a family member's death relative to election day can impact turnout in different ways, we also explore whether the turnout effects vary based on when a death occurred. We expect the opportunity cost effects to decay more quickly than the political threat effects. In other words, we expect turnout among treated voters to be higher relative to primary and secondary control voters if their household member passed away earlier in the pandemic. Moreover, because of disparities in the nature of the pandemic (Oppel et al. \protect\hyperlink{ref-Oppel2020}{2020}) we expect the treatment effects to be moderated by race and ethnicity. We therefore interact the treatment variables with voters' racioethnic characteristics.

After constructing our treatment, primary control, and secondary control groups, we implement a difference-in-differences-in-differences model at the individual-level, incorporating each individual's turnout history. These are estimated using an ordinary least squares regression. Robust standard errors are clustered at the level of the match (Abadie and Spiess \protect\hyperlink{ref-Abadie2019}{2019}).

\hypertarget{perception-and-mobilization}{%
\subsubsection*{Perception and Mobilization}\label{perception-and-mobilization}}
\addcontentsline{toc}{subsubsection}{Perception and Mobilization}

Of course, using administrative records does not allow us to directly test what we theorize to be the causal mechanism at work: namely, that voters are mobilized when they think the government failed to adequately respond to the pandemic. To test this relationship, we leverage survey data from the American National Election Studies (ANES). The ANES is administered surveyed after each federal election (with a pre-election wave in presidential election years) and asks individuals around the country a host of questions about their political beliefs.

In 2020, the ANES asked individuals about their beliefs about the novel coronavirus. Specifically, the survey asks ``Do you approve, disapprove, or neither approve nor disapprove of the way Donald Trump is handling the coronavirus (COVID-19) outbreak?'' Respondents use a 7-point scale, ranging from 1 (``Approve extremely strongly'') to 7 (``Disapprove extremely strongly''). This question allows us to test how strongly respondent's link their experience of the pandemic to the federal government's political response.

Unfortunately, the survey does not ask whether respondents lived with someone who had died from the disease, so we cannot directly test the relationship explored in the first analysis. The survey does, however, ask respondents ``How worried are you personally about getting the coronavirus (COVID-19)?'' Responses vary from 1 (``Extremely worried'') to 5 (``Not at all worried''). This allows us to measure the extent to which individuals feel a sense of threat.

If an individual is worried about contracting COVID-19 but does not link that fear to the government's response, she may not be mobilized by the pandemic. She might have lowered levels of efficacy because she feels that her safety is out of her control, or face high opportunity costs to vote. This may not be the case, however, if she links her fear of contracting the virus to the political response to the pandemic. We theorize that voters for whom the pandemic is a high-salience event and who link the pandemic to the federal response will be mobilized to participate.

To test this relationship, we run an ordinary least squares in which we interact voters' responses to the two questions discussed above. Our dependent variable measures participation in the 2020 general election. We hypothesize that high levels of fear will be associated with lower turnout. We also expect that voters with high levels of fear and strong opinions about the federal government's response reported voting at higher rates than high-fear voters will weaker opinions. Our regressions also control for other covariates known to be associated with voter turnout.

\newpage

\hypertarget{references}{%
\subsection*{References}\label{references}}
\addcontentsline{toc}{subsection}{References}

\hypertarget{refs}{}
\begin{cslreferences}
\leavevmode\hypertarget{ref-Abadie2019}{}%
Abadie, Alberto, and Jann Spiess. 2019. ``Robust Post-Matching Inference.'' \emph{Working Paper.}

\leavevmode\hypertarget{ref-Blow2020}{}%
Blow, Charles M. 2020. ``Can We Call Trump a Killer?'' \emph{The New York Times: Opinion}, June 24, 2020. \url{https://www.nytimes.com/2020/06/24/opinion/trump-coronavirus-deaths.html}.

\leavevmode\hypertarget{ref-Calia2020}{}%
Calia, Mike. 2020. ``Full Interview: President Trump Discusses Trade, Impeachment, Boeing and Elon Musk with CNBC in Davos.'' \emph{CNBC: Davos WEF}, January 22, 2020. \url{https://www.cnbc.com/2020/01/22/davos-2020-cnbcs-full-interview-with-president-trump.html}.

\leavevmode\hypertarget{ref-Campbell2003}{}%
Campbell, Andrea Louise. 2003. ``Participatory Reactions to Policy Threats: Senior Citizens and the Defense of Social Security and Medicare.'' \emph{Political Behavior} 25 (1): 29--49. \url{https://doi.org/10.1023/A:1022900327448}.

\leavevmode\hypertarget{ref-Cho2006a}{}%
Cho, Wendy K. Tam, James G. Gimpel, and Tony Wu. 2006. ``Clarifying the Role of SES in Political Participation: Policy Threat and Arab American Mobilization.'' \emph{Journal of Politics} 68 (4): 977--91. \url{https://doi.org/10.1111/j.1468-2508.2006.00482.x}.

\leavevmode\hypertarget{ref-Douglas2020}{}%
Douglas, Jason. 2020. ``As Coronavirus Surges in U.S., Some Countries Have Just About Halted It.'' \emph{Wall Street Journal: World}, July 6, 2020. \url{https://www.wsj.com/articles/as-coronavirus-surges-in-u-s-some-countries-have-just-about-halted-it-11594037814}.

\leavevmode\hypertarget{ref-Emmenegger2015}{}%
Emmenegger, Patrick, Paul Marx, and Dominik Schraff. 2015. ``Labour Market Disadvantage, Political Orientations and Voting: How Adverse Labour Market Experiences Translate into Electoral Behaviour.'' \emph{Socio-Economic Review} 13 (2): 189--213. \url{https://doi.org/10.1093/ser/mwv003}.

\leavevmode\hypertarget{ref-Fan2004}{}%
Fan, Jessie X., and Cathleen D. Zick. 2004. ``The Economic Burden of Health Care, Funeral, and Burial Expenditures at the End of Life.'' \emph{Journal of Consumer Affairs} 38 (1): 35--55. \url{https://doi.org/10.1111/j.1745-6606.2004.tb00464.x}.

\leavevmode\hypertarget{ref-Fitzpatrick2020}{}%
Fitzpatrick, Alex. 2020. ``Why the U.S. Is Losing the War on COVID-19.'' \emph{Time}, August 13, 2020. \url{https://time.com/5879086/us-covid-19/}.

\leavevmode\hypertarget{ref-Frum2020}{}%
Frum, David. 2020. ``This Is Trump's Plague Now.'' The Atlantic. June 29, 2020. \url{https://www.theatlantic.com/ideas/archive/2020/06/this-is-trumps-plague-now/613633/}.

\leavevmode\hypertarget{ref-Golden2020}{}%
Golden, Hallie. 2020. ``Coronavirus: Washington State at Center of US Outbreak as 18 Cases Confirmed.'' \emph{The Guardian: World News}, March 3, 2020. \url{https://www.theguardian.com/world/2020/mar/02/washington-state-coronavirus-outbreak-nursing-home}.

\leavevmode\hypertarget{ref-Gutierrez2019}{}%
Gutierrez, Angela, Angela X. Ocampo, Matt A. Barreto, and Gary Segura. 2019. ``Somos Más: How Racial Threat and Anger Mobilized Latino Voters in the Trump Era.'' \emph{Political Research Quarterly} 72 (4): 960--75. \url{https://doi.org/10.1177/1065912919844327}.

\leavevmode\hypertarget{ref-Hassell2017}{}%
Hassell, Hans J. G., and Jaime E. Settle. 2017. ``The Differential Effects of Stress on Voter Turnout.'' \emph{Political Psychology} 38 (3): 533--50. \url{https://doi.org/10.1111/pops.12344}.

\leavevmode\hypertarget{ref-Hobbs2014}{}%
Hobbs, William R., Nicholas A. Christakis, and James H. Fowler. 2014. ``Widowhood Effects in Voter Participation.'' \emph{American Journal of Political Science} 58 (1): 1--16. \url{https://doi.org/10.1111/ajps.12040}.

\leavevmode\hypertarget{ref-Lerman2014}{}%
Lerman, Amy E., and Vesla M. Weaver. 2014. \emph{Arresting Citizenship: The Democratic Consequences of American Crime Control}. Chicago Studies in American Politics. Chicago ; London: The University of Chicago Press.

\leavevmode\hypertarget{ref-Lu2020}{}%
Lu, Denise. 2020. ``The True Coronavirus Toll in the U.S. Has Already Surpassed 200,000.'' \emph{The New York Times: U.S.}, August 13, 2020. \url{https://www.nytimes.com/interactive/2020/08/12/us/covid-deaths-us.html}.

\leavevmode\hypertarget{ref-Margalit2019}{}%
Margalit, Yotam. 2019. ``Political Responses to Economic Shocks.'' \emph{Annual Review of Political Science} 22 (1): 277--95. \url{https://doi.org/10.1146/annurev-polisci-050517-110713}.

\leavevmode\hypertarget{ref-Marx2016}{}%
Marx, Paul, and Christoph Nguyen. 2016. ``Are the Unemployed Less Politically Involved? A Comparative Study of Internal Political Efficacy.'' \emph{European Sociological Review} 32 (5): 634--48. \url{https://doi.org/10.1093/esr/jcw020}.

\leavevmode\hypertarget{ref-McNerthney2020}{}%
McNerthney, Casey. 2020. ``Coronavirus in Washington State: A Timeline of the Outbreak Through March 2020.'' \emph{KIRO}, April 3, 2020. \url{https://www.kiro7.com/news/local/coronavirus-washington-state-timeline-outbreak/IM65JK66N5BYTIAPZ3FUZSKMUE/}.

\leavevmode\hypertarget{ref-Moon2020}{}%
Moon, Sarah. 2020. ``A Seemingly Healthy Woman's Sudden Death Is Now the First Known US Coronavirus-Related Fatality.'' \emph{CNN}, April 24, 2020. \url{https://www.cnn.com/2020/04/23/us/california-woman-first-coronavirus-death/index.html}.

\leavevmode\hypertarget{ref-Mystal2020}{}%
Mystal, Elie. 2020. ``The Deaths of 150,000 Americans Are on Trump's Hands,'' August 3, 2020. \url{https://www.thenation.com/article/society/coronavirus-trump-crimes/}.

\leavevmode\hypertarget{ref-nyt2020}{}%
\emph{New York Times: U.S.} 2020. ``Covid in the U.S.: Latest Map and Case Count,'' July 20, 2020. \url{https://www.nytimes.com/interactive/2020/us/coronavirus-us-cases.html}.

\leavevmode\hypertarget{ref-Oppel2020}{}%
Oppel, Richard A., Jr., Robert Gebeloff, K. K. Rebecca Lai, Will Wright, and Mitch Smith. 2020. ``The Fullest Look yet at the Racial Inequity of Coronavirus.'' \emph{The New York Times: U.S.}, July 5, 2020. \url{https://www.nytimes.com/interactive/2020/07/05/us/coronavirus-latinos-african-americans-cdc-data.html}.

\leavevmode\hypertarget{ref-Pacheco2015}{}%
Pacheco, Julianna, and Jason Fletcher. 2015. ``Incorporating Health into Studies of Political Behavior: Evidence for Turnout and Partisanship.'' \emph{Political Research Quarterly} 68 (1): 104--16. \url{https://doi.org/10.1177/1065912914563548}.

\leavevmode\hypertarget{ref-Piven1979}{}%
Piven, Frances Fox, and Richard A. Cloward. 1979. \emph{Poor People's Movements: Why They Succeed, How They Fail}. New York: Vintage books.

\leavevmode\hypertarget{ref-whitehouse2020}{}%
``Remarks by President Trump at a USMCA Celebration with American Workers \textbar{} Warren, MI.'' 2020. The White House. January 30, 2020. \url{https://www.whitehouse.gov/briefings-statements/remarks-president-trump-usmca-celebration-american-workers-warren-mi/}.

\leavevmode\hypertarget{ref-Rosenstone1982}{}%
Rosenstone, Steven J. 1982. ``Economic Adversity and Voter Turnout.'' \emph{American Journal of Political Science} 26 (1): 25--46. \url{https://doi.org/10.2307/2110837}.

\leavevmode\hypertarget{ref-Rudolph2000}{}%
Rudolph, Thomas J., Amy Gangl, and Dan Stevens. 2000. ``The Effects of Efficacy and Emotions on Campaign Involvement.'' \emph{The Journal of Politics} 62 (4): 1189--97. \url{https://doi.org/10.1111/0022-3816.00053}.

\leavevmode\hypertarget{ref-Saletan2020}{}%
Saletan, William. 2020. ``How Trump Killed Tens of Thousands of Americans.'' Slate Magazine. August 9, 2020. \url{https://slate.com/news-and-politics/2020/08/trump-coronavirus-deaths-timeline.html}.

\leavevmode\hypertarget{ref-Sandell2005}{}%
Sandell, Julianna, and Eric Plutzer. 2005. ``Families, Divorce and Voter Turnout in the US.'' \emph{Political Behavior} 27 (2): 133--62. \url{https://doi.org/10.1007/s11109-005-3341-9}.

\leavevmode\hypertarget{ref-Schlozman1979}{}%
Schlozman, Kay Kehman, and Sidney Verba. 1979. \emph{Injury to Insult: Unemployment, Class, and Political Response}. Harvard University Press. \url{http://books.google.com?id=m6XTdRlx3Q8C}.

\leavevmode\hypertarget{ref-Sekhon2011}{}%
Sekhon, Jasjeet S. 2011. ``Multivariate and Propensity Score Matching Software with Automated Balance Optimization: The Matching Package for R.'' \emph{Journal of Statistical Software} 42 (1): 1--52. \url{https://doi.org/10.18637/jss.v042.i07}.

\leavevmode\hypertarget{ref-Shear2020}{}%
Shear, Michael D., Noah Weiland, Eric Lipton, Maggie Haberman, and David E. Sanger. 2020. ``Inside Trump's Failure: The Rush to Abandon Leadership Role on the Virus.'' \emph{The New York Times: U.S.}, July 18, 2020. \url{https://www.nytimes.com/2020/07/18/us/politics/trump-coronavirus-response-failure-leadership.html}.

\leavevmode\hypertarget{ref-Walker2020}{}%
Walker, Hannah L. 2020. ``Targeted: The Mobilizing Effect of Perceptions of Unfair Policing Practices.'' \emph{The Journal of Politics} 82 (1): 119--34. \url{https://doi.org/10.1086/705684}.

\leavevmode\hypertarget{ref-Wan2020}{}%
Wan, William. 2020. ``WHO Declares a Pandemic of Coronavirus Disease Covid-19.'' \emph{Washington Post}, March 11, 2020. \url{https://www.washingtonpost.com/health/2020/03/11/who-declares-pandemic-coronavirus-disease-covid-19/}.

\leavevmode\hypertarget{ref-Weaver2010}{}%
Weaver, Vesla M., and Amy E. Lerman. 2010. ``Political Consequences of the Carceral State.'' \emph{American Political Science Review} 104 (4): 817--33. \url{https://doi.org/10.1017/S0003055410000456}.

\leavevmode\hypertarget{ref-White2019a}{}%
White, Ariel. 2019. ``Family Matters? Voting Behavior in Households with Criminal Justice Contact.'' \emph{American Political Science Review} 113 (2): 607--13. \url{https://doi.org/10.1017/S0003055418000862}.

\leavevmode\hypertarget{ref-White2016}{}%
---------. 2016. ``When Threat Mobilizes: Immigration Enforcement and Latino Voter Turnout.'' \emph{Political Behavior} 38 (2): 355--82. \url{https://doi.org/10.1007/s11109-015-9317-5}.

\leavevmode\hypertarget{ref-Wyatt1999}{}%
Wyatt, Gwen K., Laurie Friedman, Charles W. Given, and Barbara A. Given. 1999. ``A Profile of Bereaved Caregivers Following Provision of Terminal Care.'' \emph{Journal of Palliative Care} 15 (1): 13--25. \url{https://doi.org/10.1177/082585979901500103}.
\end{cslreferences}

\end{document}
